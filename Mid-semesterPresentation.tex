\documentclass{beamer}


\usepackage{comment}
\usepackage[utf8]{inputenx} % For æ, ø, å
\usepackage{babel}          % Automatic translations
\usepackage{csquotes}       % Quotation marks
\usepackage{microtype}      % Improved typography
\usepackage{amssymb}        % Mathematical symbols
\usepackage{mathtools}      % Mathematical symbols
\usepackage[absolute, overlay]{textpos} % Arbitrary placement
\setlength{\TPHorizModule}{\paperwidth} % Textpos units
\setlength{\TPVertModule}{\paperheight} % Textpos units
\usepackage{tikz}
\usetikzlibrary{overlay-beamer-styles}  % Overlay effects for TikZ


\author{A. Eren Uyanık}
\title{Classical Fourier Series}
\subtitle{Mid-semester presentation}
\newcommand{\univ}{Bilkent University}
\newcommand{\advisor}{Aurelian Gheondea}



\setbeamertemplate{title page}{%
  \vspace{2cm}
  \begin{center}
    \usebeamerfont{title}\inserttitle\par
    \usebeamerfont{subtitle}\usebeamercolor[fg]{subtitle}\insertsubtitle\par
    \vspace{1cm}
    \usebeamerfont{author}\insertauthor\par
    \vspace{0.5cm} % Adjust the spacing as needed
    \usebeamerfont{institute}\univ\par % Add the university
    \vspace{0.3cm} % Adjust the spacing as needed
    \usebeamerfont{institute}\advisor\par % Add the advisor
  \end{center}
  \vfill
}




\begin{document}

\begin{frame}[plain]
  \titlepage
\end{frame}

\section{Overview}


% Use
%
%     \begin{frame}[allowframebreaks]{Title}
%
% if the TOC does not fit one frame.
\begin{frame}{Table of contents}
    \tableofcontents[currentsection]
\end{frame}


\section{History and Motivation}



\begin{frame}{History and Motivation}
\small
\begin{enumerate}
    \item Propagation of heat.
    \item Birth of the definitions of ...
    \begin{itemize}
        \item Functions
        \item Types of convergence (absolute, uniform, pointwise)
        \item Riemann and Lebesque integrals
        \item (Cantor's) Set theory
    \end{itemize}
\end{enumerate}
\end{frame}




\section{Introduction}
%\SectionPage


\begin{frame}{Trigonometric Series}
\small
    \begin{definition}
        A trigonometric series $S$ is defined as
        \[
        S(x) = \frac{a_0}{2} +  \sum_{k = 1}^{\infty} (a_k  \cos(kx) + b_k  \sin(kx))
        \]
        Partial sums, $S_N$, are called as trigonometric polynomial.

    \end{definition}
    \begin{definition}[Orthogonality]
        Two (integrable) functions $f, g$ are called orthogonal if,
        \[
        \int_{-\pi}^{\pi}f.g = 0
        \]
    \end{definition}

\end{frame}
\begin{frame}{Some Orthogonalities}
\small
    Before giving the definition of \textit{Fourier Series}, we give some useful identities.

    \[
    \int_{-\pi}^{\pi} \cos(mx)cos(nx) = \begin{cases}
                                        0 & m \neq n \\
                                      \pi &  m = n \neq 0\\
                                     2\pi & m = n = 0
                                        \end{cases}
    \]
    \[
    \int_{-\pi}^{\pi} \sin(mx)sin(nx) = \begin{cases}
                                        0 & m \neq n \\
                                      \pi &  m = n \neq 0\\
                                        0 & m = n = 0
                                        \end{cases}
    \]
    \[
    \int_{-\pi}^{\pi} \sin(mx)cos(nx) = 0
    \]
\end{frame}
\begin{frame}[shrink=9]{Fourier Series}
\small
    \begin{definition}
        Let \(f\) be an integrable function on [$\pi, \pi$]. Then, Fourier series of \(f\), namely $S($\(f\)) is defined as
    \[
    S(f)(x) = \frac{a_0(f)}{2} +  \sum_{k = 1}^{\infty} (a_k(f)  \cos(kx) + b_k(f)  \sin(kx))
    \]
    where the coefficients $a_k(f)$ and $b_k(f)$ are defined as

    \[
    a_k(f) = \frac{2}{\pi}\int_{-\pi}^{\pi} f(x) \cos(kx)\,dx
    \]
    \[
    b_k(f) = \frac{2}{\pi}\int_{-\pi}^{\pi} f(x) \sin(kx)\,dx
    \]
    \end{definition}

    \begin{theorem}[Fourier]
    If a trigonometric series
    \[
    S := \frac{a_0}{2} +  \sum_{k = 1}^{\infty} (a_k  \cos(kx) + b_k  \sin(kx))
    \]
   converges uniformly to a function \(f\), the $S$ is the Fourier series of \(f\), i.e. $S(f)$.
\end{theorem}
\end{frame}

\section{Summability}


\begin{frame}{Cesaro Sums}
    We ask whether the Fourier Series of $f$ converge to $f$ or not. It turns out to be really difficult question. Here comes a weaker version of convergence.
    \begin{definition}[Cesaro Summable]
        A series $\sum_{k=0}^{\infty} a_k$ with partial sums $S_N$ is said to be \textit{Cesaro summable} to $L$ if \textit{Cesaro means}
    \[
    \sigma_N \vcentcolon =  \frac{S_0 + S_1 + \dots + S_N }{N+1}
    \]
    converges to $L$ as $N \longrightarrow \infty$.
    \end{definition}


\end{frame}

    \begin{frame}{Two Kernels}
    \small
    In general, we will not deal with the original definitions but the equivalent versions of them. Two kernels defined here will make theorems a lot easier to prove via equivalences given as convolutions.
    \begin{definition}[Kernels]
        Let $N$ be a nonnegative integer.
\begin{enumerate}
  \item[(i)] The Dirichlet kernel of order $N$ is the function defined, for each $x \in \mathbb{R}$, by
  \[
  D_N(x) = \begin{cases}
            \frac{1}{2} & \text{if } N = 0, \\
            \frac{1}{2}  + \sum_{k=1}^{N} \cos(kx) & \text{if } N \in \mathbb{N}.
           \end{cases}
  \]

  \item[(ii)] The Fejer kernel of order $N$ is the function defined, for each $x \in \mathbb{R}$, by
  \[
  K_N(x) = \begin{cases}
            \frac{1}{2} & \text{if } N = 0, \\
            \frac{1}{2} + \sum_{k=1}^{N}\left( 1 - \frac{k}{N + 1} \right)\cos(kx) & \text{if } N \in \mathbb{N}.
           \end{cases}
  \]
\end{enumerate}
    \end{definition}

\end{frame}




\begin{frame}{Semi-Convergence}
\small
\begin{theorem}[Fejer]
    Suppose that $f: \mathbb{R} \to \mathbb{R}$ is periodic and integrable on each compact interval.

        1. If \begin{equation*}
             L = \lim_{h \to 0} \frac{f(x_0 + h) + f(x_0 - h)}{2}
        \end{equation*}

        exists for some $x_0 \in \mathbb{R}$, then $\sigma_Nf(x_0) \to L$ as $N \to \infty$.   \\
        2. If \(f\) is continuous in some closed interval $I$, then $\sigma_Nf \to f$ uniformly on $I$.

\end{theorem}

An important theorem can be proven using \textit{Fejer Theorem}:
\begin{theorem}[Weierstrass Approximation Theorem]
Let $[a, b]$ be a closed bounded interval, and suppose that $f : [a, b] \rightarrow \mathbb{R}$ is continuous on $[a, b]$. Given $c > 0$, there exists a (classical) polynomial $P$ on $\mathbb{R}$ such that $|f(x) - P(x)| < c$ for all $x \in [a, b]$.

\end{theorem}
\end{frame}



\section{Fourier Coefficients}


\begin{frame}{Convergence to 0}
    \small
    We know that if a series converges, then its terms converge to zero. Here, we get a reason to hope $Sf$ will converge to $f$.
    \begin{theorem}[Riemann-Lebesque Lemma]
        If $f$ is integrable on $[-\pi,\pi]$, then
        \[
        \lim_{k \to \infty}a_k(f)  = \lim_{k \to \infty}b_k(f) = 0
        \]
    \end{theorem}
    This lemma can be proved via the famous \textit{Parseval's identity}.
    \begin{theorem}[Parseval's Identity]
        If $f: \mathbb{R} \to \mathbb{R}$ is periodic and continuous, then
        \[
        \frac{a_0(f)^2}{2} + \sum_{k = 1}^{\infty}(a_k(f)^2 + b_k(f)^2) = \frac{1}{\pi}\int_{-\pi}^{\pi}f^2
        \]
    \end{theorem}

\end{frame}
\begin{frame}{Speed of convergence to 0}
    Depending of the properties of a given function, we can estimate how fast the coefficients will converge to 0.
    \begin{theorem}
        Given a function \( f : \mathbb{R} \to \mathbb{R} \) and \( j \in \mathbb{N} \). If \( f^{(j)} \) exists and is integrable on the interval \([-\pi, \pi]\) and \( f^{(l)} \) is periodic for each \( 0 \leq l < j \), then
\[
\lim_{{k \to \infty}}k^ja_k(f)  = \lim_{{k \to \infty}} k^jb_k(f) = 0.
\]

    \end{theorem}
\end{frame}



\section{Next}
\begin{frame}{To be continued...}
    Until the end of the semester, we plan to study:
    \begin{enumerate}
        \item Convergence of Fourier Series
        \begin{itemize}
            \item Tauber theorems (and Hardy's theorem as Tauber type)
            \item Dirichlet - Jordan theorem
        \end{itemize}
        \item Uniqueness
        \begin{itemize}
            \item second symmetric derivative
            \item Riemann \& Cantor theorems
            \item Contor-Lebesque lemma
        \end{itemize}
    \end{enumerate}
\end{frame}


\section{Summary}
\begin{frame}{Summary}
    \small
    \begin{enumerate}
        \item Influences on modern analysis
        \item Trigonometric polynomials and orthogonality properties
        \item Definition of Fourier Series
        \item Cesaro summability of Fourier Series and semi-convergence
        \item Behaviour of Fourier coefficients
        \item Parseval's identity
        \item What is next?
    \end{enumerate}
\end{frame}



\section{References}


\begin{frame}[allowframebreaks]{References}
\small
    \begin{thebibliography}{}

        % Article is the default.
        %\setbeamertemplate{bibliography item}[book]

        \bibitem{Wade}
        Wade, W.R.
        \newblock \emph{Introduction to Analysis}.
        \newblock Pearson, 2004.

        \bibitem{History}
        Niels Jacob and Kristian P. Evans
        \newblock \emph{A course in Analysis}.
        \newblock World Scientific, 2018.


    \end{thebibliography}
\end{frame}


\end{document}