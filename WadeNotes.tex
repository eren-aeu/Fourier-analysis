\documentclass{article}
\usepackage{graphicx} % Required for inserting images

\title{Fourier analysis}
\author{eren}
\date{July 2023}
\usepackage{amsmath,amssymb, amsthm, mathtools}

\theoremstyle{remark}
\newtheorem*{remark}{Remark}
\theoremstyle{lemma}
\newtheorem{lemma}{Lemma}
\theoremstyle{example}
\newtheorem*{example}{Example}



\theoremstyle{definition}
\newtheorem{definition}{Definition}
\newtheorem{theorem}{Theorem}
\begin{document}

\maketitle

\section{What is Fourier Analysis?}
\section{Why Fourier Analysis?}

\section{Introduction}
\begin{definition}[Fourier series]\footnotemark{}
\footnotetext{Although I do not like the "devil's triangle", definition + theorem + proof, for a fast introduction I hope it is okay.}
    Let \(f\) be an integrable function on [$\pi, \pi$]. Then, Fourier series of \(f\), namely $S($\(f\)) is defined as
    \[
    S(f)(x) = \frac{a_0(f)}{2} +  \sum_{k = 1}^{\infty} (a_k(f)  \cos(kx) + b_k(f)  \sin(kx))
    \]

    The partial sums $S_N($\(f\)) are defined as
    \[
    S_N(f)(x) = \frac{a_0(f)}{2} +  \sum_{k = 1}^{N} (a_k(f)  \cos(kx) + b_k(f)  \sin(kx))
    \]

    where the coefficients $a_k(f)$ and $b_k(f)$ are defined as

    \[
    a_k(f) = \frac{2}{\pi}\int_{-\pi}^{\pi} f(x) \cos(kx)\,dx
    \]
    \[
    b_k(f) = \frac{2}{\pi}\int_{-\pi}^{\pi} f(x) \sin(kx)\,dx
    \]
\end{definition}
\begin{theorem}[Fourier]
    If a trigonometric series
    \[
    S := a_0  / 2 +  \sum_{k = 1}^{N} (a_k  \cos(kx) + b_k  \sin(kx))
    \]
   converges uniformly to a function \(f\), the $S$ is the Fourier series of \(f\), i.e. $S(f)$.
\end{theorem}

    Here comes an important and a modern concept, orthogonality\footnotemark{}. We say two functions $f$ and $g$ are orthogonal (in our context, Fourier analysis) if $\int_{-\pi}^{\pi} f(x)g(x)\,dx = 0$. A careful observer may notice after some time spend on struggling with the proof that these three "orthogonalities" hold:
    \[
    \int_{-\pi}^{\pi} \cos(mx)cos(nx) = \begin{cases}
                                        0 & m \neq n \\
                                      \pi &  m = n \neq 0\\
                                     2\pi & m = n = 0
                                        \end{cases}
    \]
    \[
    \int_{-\pi}^{\pi} \sin(mx)sin(nx) = \begin{cases}
                                        0 & m \neq n \\
                                      \pi &  m = n \neq 0\\
                                        0 & m = n = 0
                                        \end{cases}
    \]
    \[
    \int_{-\pi}^{\pi} \sin(mx)cos(nx) = 0
    \]

    \footnotetext{As the name suggests, it is a generalisation of the concept "being perpendicular}
\begin{proof}[Proof idea]
    If we multiply each side with $\cos(kx)$ and integrate, all the terms except the $a_k \cos(kx)$ will disappear. From here $a_k = a_k(f)$ follows easily. \hfill
    \indent Notice the necessity of having "uniform" convergence. In order to be able to use the power of orthogonality, we should "distribute" the integral to each term of the series which can only be done under the assumption of uniform convergence.
\end{proof}
    Now, two central questions in Fourier analysis is to be stated:

\textbf{Convergence Question.} Given a function \( f : \mathbb{R} \to \mathbb{R} \), periodic on \( \mathbb{R} \) and integrable on \( [-\pi, \pi] \), does the Fourier series of \( f \) converge to \( f \)?

\textbf{Uniqueness Question.} If a trigonometric series \( S \) converges to some function \( f \) integrable on \( [-\pi, \pi] \), is S the Fourier series of \(f\)?

Uniqueness question is answered if the convergence is uniform. We will try to answer these two fundemental questions in the next sections. Now comes the two important definitions which will play an important role in the next chapter.\footnotemark{}
\footnotetext{You may say "Then why given here?" You are right.}

\textbf{Definition.} Let $N$ be a nonnegative integer.
\begin{enumerate}
  \item[(i)] The Dirichlet kernel of order $N$ is the function defined, for each $x \in \mathbb{R}$, by
  \[
  D_N(x) = \begin{cases}
            \frac{1}{2} & \text{if } N = 0, \\
            \frac{1}{2}  + \sum_{k=1}^{N} \cos(kx) & \text{if } N \in \mathbb{N}.
           \end{cases}
  \]

  \item[(ii)] The Fejer kernel of order $N$ is the function defined, for each $x \in \mathbb{R}$, by
  \[
  K_N(x) = \begin{cases}
            \frac{1}{2} & \text{if } N = 0, \\
            \frac{1}{2} + \sum_{k=1}^{N}\left( 1 - \frac{k}{N + 1} \right)\cos(kx) & \text{if } N \in \mathbb{N}.
           \end{cases}
  \]
\end{enumerate}

\begin{remark}
    If $N$ is a nonnegative integer, then
\[
K_N(x) = \frac{D_0(x) + D_1(x) + \dots + D_N(x)}{N+1}
\]
for all $x \in \mathbb{R}$
\end{remark}

\begin{theorem}
    If $x$ cannot be written in the form $2k\pi$ for any $k \in \mathbb{Z}$, then the following equalities hold

    \[
    D_N(x) = \frac{\sin(N + \frac{1}{2})x}{2\sin(\frac{x}{2})}
    \]
    \[
    K_N(x) = \frac{2}{N+1}(\frac{\sin(\frac{N + 1}{2})x}{2\sin(\frac{x}{2})})^2
    \]
\end{theorem}
\begin{proof}[Proof idea] We try to express  $D_N(x)\sin(\frac{x}{2}$ in a different form. Here we remember (after a while of course) the forgotten high school formula,
\[
\cos(\alpha)\sin(\beta) = \sin(\alpha + \beta) - \sin(\alpha - \beta)
\]
    Then, the result is a telescoping sum yielding the desired expression.
\end{proof}



     \begin{remark}
         Prove that if \(f\) is integrable, then
    \[
    S_N(f)(x) = \frac{1}{\pi}\int_{-\pi}^{\pi}f(t)D_N(x-t)\,dt
    \]
     \end{remark}
    \begin{proof}
        we start by looking at an arbitrary term of $S_Nf(x)$,

        \begin{align*}
            a_k\cos(kx) + b_k\sin(kx) &= \frac{1}{\pi}\int_{-\pi}^{\pi}\cos(kt)f(t)dt \cdot \cos(kx) + \frac{1}{\pi}\int_{-\pi}^{\pi}\sin(kt)f(t)dt \cdot \sin(kx)\\
        &= \frac{1}{\pi}\int_{-\pi}^{\pi}f(t)(\cos(kt)\cos(kx) + \sin(kt)\sin(kx))dt\\
        &= \frac{1}{\pi}\int_{-\pi}^{\pi}f(t)\cos((x-t)k)dt\\
        \end{align*}
        Now, we add these terms up,
        \begin{align*}
            S_N(x) &=\frac{a_0}{2} + \sum_{k=1}^{N} (a_k\cos(kx) + b_k\sin(kx))\\
            &= \frac{a_0}{2} + \sum_{k=1}^{N}(\frac{1}{\pi}\int_{-\pi}^{\pi}f(t)\cos((x-t)k)dt) \\
            &= \frac{1}{\pi}\int_{-\pi}^{\pi}f(t) (\frac{a_0}{2} + \sum_{k=1}^{N}\cos((x-t)k)dt)) \\
            &= \frac{1}{\pi}\int_{-\pi}^{\pi}f(t)D_N(x-t)dt \qedhere
        \end{align*}

    \end{proof}

     \begin{example}
     Let \(f\) be a function defined as
    \[
    f(x) = \begin{cases}
        \frac{x}{|x|} & x \neq 0\\
        0             & x = 0
    \end{cases}
    \]
    \textbf{(a)} Compute the Fourier coefficients of $f$.

    \textbf{(b)} Prove that
    \[
    (S_{2N} f)(x) = \frac{2}{\pi} \int_{-\pi}^{\pi} \frac{\sin(2Nt)}{\sin(t)} \,dt
    \]
    for $x \in [-\pi,\pi]$ and $N \in \mathbb{N}$.

    \textbf{(c) [Gibbs's phenomenon].} Prove that
    \[
    \lim_{N \to \infty}(S_{2N} f)(\frac{\pi}{2N}) = -\frac{2}{\pi}\int_{-\pi}^{\pi} \frac{\sin(t)}{t} dt \approx 1.179
    \]
    as $N \to \infty$.
     \end{example}


    \begin{proof}[Hint]
        \textbf{(c)} Use uniform convergence to interchange the limit and integral.
    \end{proof}

\section{Summability of Fourier Series}
Continuous functions are known to be "well-behaving" functions in many contexts for many mathematicians. But when it comes to Fourier analysis, the basic (and basis) "Convergence question" becomes very difficult to answer even for continuous functions, of course better than not-continuous ones. Here we replace "convergence of functions" with a simpler kind of "convergence", being summable, and show that Fourier series of any continuous periodic function \(f\) is uniformly summable to \(f\).

\begin{definition}
    A series $\sum_{k=0}^{\infty} a_k$ with partial sums $S_N$ is said to be \textit{Cesaro summable} to $L$ if \textit{Cesaro means}
    \[
    \sigma_N \vcentcolon =  \frac{S_0 + S_1 + \dots + S_N }{N+1}
    \]
    converges to $L$ as $N \longrightarrow \infty$.
\end{definition}
Following remarks shows that summability is a generalisation of convergence.
\begin{remark}
    If $\sum_{k=0}^{\infty} a_k$ converges to $L < \infty$ then it is Cesaro summable to $L$.
\end{remark}
Note that the converse of the remark is not true, as an example take $\sum_{k = 0}^{\infty} (-1) ^ k$.\par
Now, we weaken our \textit{Convergence Question}: \\
\textbf{Summability Question} Given a function $f: \mathbb{R} \xrightarrow{} \mathbb{R} $ periodic on $\mathbb{R}$ and integrable on $[-\pi, \pi]$, is $Sf$ Cesaro summable to \(f\)?\par
We give here a lemma which will make $\sigma_N f$ easy to deal with (especially when we try to estimate $|\sigma_N f - f|$).

\begin{lemma}
    Let $f: \mathbb{R} \xrightarrow{} \mathbb{R} $ periodic on $\mathbb{R}$ and integrable on $[-\pi, \pi]$. Then,
    \[
    \sigma_Nf(x) = \frac{1}{\pi} \int_{-\pi}^{\pi} K_N(t)f(x-t)dt
    \]
    for all $N \in \mathbb{N}$ and for all $ x \in \mathbb{R}$.\footnotemark{}
    \footnotetext{This may remind you the convolution and you are right!}
\end{lemma}

\begin{proof}
    \begin{align*}
    \sigma_Nf(x) &= \frac{S_0f + S_1f + \dots + S_Nf}{N+1}\\
                 &= \frac{\sum_{k = 0}^{N}S_kf}{N+1} \\
                 &= \frac{\sum_{k = 0}^{N}\frac{1}{\pi}\int_{-\pi}^{\pi}D_k(t)f(x-t)dt}{N+1} &&\text{(see exercise 1.1)}\\
                 &= \frac{1}{\pi}\int_{-\pi}^{\pi}f(x-t)\frac{\sum_{k = 0}^{N}D_k}{N+1}dt &&\text{(interchange sum \& integral for finite series)}\\
                 &= \frac{1}{\pi}\int_{-\pi}^{\pi}f(x-t)K_N(t)dt &&\text{(see the remark for $D_N$ and $K_N$)}\qedheres
    \end{align*}
\end{proof}
\par
To go further, we need to investigate \textit{Fejer kernel}s more. As will be seen, they act similar to the so-called \textit{Dirac-delta} ($\delta$) function.
\begin{lemma}
    For each natural number $N$,
    \begin{align}
        K_N(t) &\geq 0 for all t \in \mathbb{R},\\
        \intertext{and}
        &\frac{1}{\pi}\int_{-\pi}^{\pi} K_N(t)dt = 1,\\
        \intertext{for each  $\pi > \delta > 0$,}
        &\lim_{N \to \infty} \int_{\delta}^{\pi}|K_N(t)|dt = 0.
    \end{align}
\end{lemma}
\begin{proof}{Proof idea.}
    For the first one, use Theorem 2 for $x \neq 2\pi k$. For second, use the definition of $K_N$ and for the third, observe that $\sin(\frac{t}{2}) \geq \sin(\frac{\delta}{2})$ for all $t > \delta$ and use Theorem 2.
\end{proof}

Now comes the theorem we have been giving lemma after lemma for just to be able to prove more easily, \textit{Fejer Theorem};

\begin{theorem}{[Fejer]}
    Suppose that $f: \mathbb{R} \to \mathbb{R}$ is periodic and integrable on each compact interval.\footnotemark{} \footnotetext{Ask}

        1. If \begin{equation}
             L = \lim_{h \to 0} \frac{f(x_0 + h) + f(x_0 - h)}{2}
        \end{equation}

        exists for some $x_0 \in \mathbb{R}$, then $\sigma_Nf(x_0) \to L$ as $N \to \infty$.   \\
        2. If \(f\) is continuous in some closed interval $I$, then $\sigma_Nf \to f$ uniformly on $I$.

\end{theorem}

\end{document}
