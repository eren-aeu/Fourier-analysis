% Senior Project Student: Adem eren uyanık
% Department of Mathematics, Bilkent University
% Aurelian Gheondea, Senior Project Coordinator

\documentclass[12pt]{amsart}
\usepackage{amsmath,amssymb, mathtools}
\usepackage{amsthm}
% The package can be used temporarily only to show the cross-reference labels
%\usepackage{showkeys}
% This paragraph resets the sizes of the printed area of the page
\hoffset -1.5cm
\voffset -1cm
\textwidth 15.5truecm
\textheight 22.5truecm
% This paragraph defines the macros for theorems and the like
\newtheorem{theorem}{Theorem}[section]
\newtheorem{proposition}[theorem]{Proposition}
\newtheorem{corollary}[theorem]{Corollary}
\newtheorem{lemma}[theorem]{Lemma}
% This paragraph defines the macros for definitions and the like
\theoremstyle{definition}
\newtheorem{definition}[theorem]{Definition}
\newtheorem{example}[theorem]{Example}
\newtheorem{conjecture}[theorem]{Conjecture}
\newtheorem{remark}[theorem]{Remark}
\newtheorem{remarks}[theorem]{Remarks}
% The following paragraph writes the equation numbers with two counters,
% the first is the section number and the second resets within the section.
\makeatletter
\@addtoreset{equation}{section}
\makeatother
\renewcommand\theequation{\arabic{section}.\arabic{equation}}
% The next paragraph defines macros for special roman letters to be used
\newcommand{\CC}{{\mathbb C}} % the set of complex numbers
\newcommand{\NN}{{\mathbb N}} % the set of natural numbers
\newcommand{\QQ}{{\mathbb Q}} % the set of rational numbers
\newcommand{\ZZ}{{\mathbb Z}} % the set of integer numbers
\newcommand{\DD}{{\mathbb D}} % the unit disk
\newcommand{\RR}{{\mathbb R}} % the set of real numbers
\newcommand{\TT}{{\mathbb T}} % the unit circle (the one dimensional torus)
% The next paragraph defines macros for caligraphic letters
\newcommand{\cA}{{\mathcal A}}
\newcommand{\cB}{{\mathcal B}}
\newcommand{\cC}{{\mathcal C}}
\newcommand{\cD}{{\mathcal D}}
\newcommand{\cE}{{\mathcal E}}
\newcommand{\cF}{{\mathcal F}}
\newcommand{\cG}{{\mathcal G}}
\newcommand{\cH}{{\mathcal H}}
\newcommand{\cJ}{{\mathcal J}}
\newcommand{\cK}{{\mathcal K}}
\newcommand{\cL}{{\mathcal L}}
\newcommand{\cM}{{\mathcal M}}
\newcommand{\cN}{{\mathcal N}}
\newcommand{\cP}{{\mathcal P}}
\newcommand{\cQ}{{\mathcal Q}}
\newcommand{\cR}{{\mathcal R}}
\newcommand{\cS}{{\mathcal S}}
\newcommand{\cT}{{\mathcal T}}
\newcommand{\cU}{{\mathcal U}}
\newcommand{\cV}{{\mathcal V}}
\newcommand{\cW}{{\mathcal W}}
\newcommand{\cZ}{{\mathcal Z}}
% The next paragraph defines special macros
\newcommand{\iac}{\mathrm{i}} % the imaginary i
\newcommand{\de}{\operatorname{d}} % the differential
\newcommand{\deriv}{\operatorname{D}}
\providecommand{\AMS}{$\mathcal{A}$\kern-.1667em%
\lower.25em\hbox{$\mathcal{M}$}\kern-.125em$\mathcal{S}$}
% Use of macros should confine the general rules for AMS-LaTeX. It is
% recommended to use \newcommand and \renewcommand instead of the TeX
% primitive \def
% Next is an example of macro I designed for numbering items that are
% different than the available ``itemize, description'' environments.
% You can use it if you want!
\newcommand{\nr}[1]{\vspace{0.1ex}\noindent\hspace*{12mm}\llap{\textup{(#1)}}}
% Topmatter produces the title, author, abstract, etc.
\begin{document}

\title[Classical Fourier Series]{Classical Fourier Series}


\author{A. Eren Uyanık}

%\address{Department of Mathematics, Bilkent University, 06800 %Bilkent, Ankara,
%  Turkey}
%\email{eren.uyanik@ug.bilkent.edu.tr}

\begin{abstract}

Abstract will be added
\end{abstract}
\maketitle

% Body of the project

\section*{Overview} % The * is needed if you do not want this section
% numbered; otherwise remove the * and this will be numbered by 1.

Some history and the overview of subject  will be given here. Also where to add the advisor?


\section{Introduction}
\begin{definition}[Fourier series]\footnotemark{}
\footnotetext{Although I do not like the "devil's triangle", definition + theorem + proof, for a fast introduction I hope it is okay.}
    Let \(f\) be an integrable function on [$\pi, \pi$]. Then, Fourier series of \(f\), namely $S($\(f\)) is defined as
    \[
    S(f)(x) = \frac{a_0(f)}{2} +  \sum_{k = 1}^{\infty} (a_k(f)  \cos(kx) + b_k(f)  \sin(kx))
    \]

    The partial sums $S_N($\(f\)) are defined as
    \[
    S_N(f)(x) = \frac{a_0(f)}{2} +  \sum_{k = 1}^{N} (a_k(f)  \cos(kx) + b_k(f)  \sin(kx))
    \]

    where the coefficients $a_k(f)$ and $b_k(f)$ are defined as

    \[
    a_k(f) = \frac{2}{\pi}\int_{-\pi}^{\pi} f(x) \cos(kx)\,dx
    \]
    \[
    b_k(f) = \frac{2}{\pi}\int_{-\pi}^{\pi} f(x) \sin(kx)\,dx
    \]
\end{definition}
\begin{theorem}[Fourier]
    If a trigonometric series
    \[
    S := a_0  / 2 +  \sum_{k = 1}^{\infty} (a_k  \cos(kx) + b_k  \sin(kx))
    \]
   converges uniformly to a function \(f\), the $S$ is the Fourier series of \(f\), i.e. $S(f)$.
\end{theorem}

    Here comes an important and a modern concept, orthogonality\footnotemark{}. We say two functions $f$ and $g$ are orthogonal (in our context, Fourier analysis) if $\int_{-\pi}^{\pi} f(x)g(x)\,dx = 0$. A careful observer may notice after some time spend on struggling with the proof that these three "orthogonalities" hold:
    \[
    \int_{-\pi}^{\pi} \cos(mx)cos(nx) = \begin{cases}
                                        0 & m \neq n \\
                                      \pi &  m = n \neq 0\\
                                     2\pi & m = n = 0
                                        \end{cases}
    \]
    \[
    \int_{-\pi}^{\pi} \sin(mx)sin(nx) = \begin{cases}
                                        0 & m \neq n \\
                                      \pi &  m = n \neq 0\\
                                        0 & m = n = 0
                                        \end{cases}
    \]
    \[
    \int_{-\pi}^{\pi} \sin(mx)cos(nx) = 0
    \]

    \footnotetext{As the name suggests, it is a generalisation of the concept "being perpendicular}
\begin{proof}[Proof idea]
    If we multiply each side with $\cos(kx)$ and integrate, all the terms except the $a_k \cos(kx)$ will disappear. From here $a_k = a_k(f)$ follows easily. \hfill
    \indent Notice the necessity of having "uniform" convergence. In order to be able to use the power of orthogonality, we should "distribute" the integral to each term of the series which can only be done under the assumption of uniform convergence.
\end{proof}
    Now, two central questions in Fourier analysis is to be stated:

\textbf{Convergence Question.} Given a function \( f : \mathbb{R} \to \mathbb{R} \), periodic on \( \mathbb{R} \) and integrable on \( [-\pi, \pi] \), does the Fourier series of \( f \) converge to \( f \)?

\textbf{Uniqueness Question.} If a trigonometric series \( S \) converges to some function \( f \) integrable on \( [-\pi, \pi] \), is S the Fourier series of \(f\)?

Uniqueness question is answered if the convergence is uniform. We will try to answer these two fundemental questions in the next sections. Now comes the two important definitions which will play an important role in the next chapter.\footnotemark{}
\footnotetext{You may say "Then why given here?" You are right.}

\textbf{Definition.} Let $N$ be a nonnegative integer.
\begin{enumerate}
  \item[(i)] The Dirichlet kernel of order $N$ is the function defined, for each $x \in \mathbb{R}$, by
  \[
  D_N(x) = \begin{cases}
            \frac{1}{2} & \text{if } N = 0, \\
            \frac{1}{2}  + \sum_{k=1}^{N} \cos(kx) & \text{if } N \in \mathbb{N}.
           \end{cases}
  \]

  \item[(ii)] The Fejer kernel of order $N$ is the function defined, for each $x \in \mathbb{R}$, by
  \[
  K_N(x) = \begin{cases}
            \frac{1}{2} & \text{if } N = 0, \\
            \frac{1}{2} + \sum_{k=1}^{N}\left( 1 - \frac{k}{N + 1} \right)\cos(kx) & \text{if } N \in \mathbb{N}.
           \end{cases}
  \]
\end{enumerate}

\begin{remark}
    If $N$ is a nonnegative integer, then
\[
K_N(x) = \frac{D_0(x) + D_1(x) + \dots + D_N(x)}{N+1}
\]
for all $x \in \mathbb{R}$
\end{remark}

\begin{theorem}
    If $x$ cannot be written in the form $2k\pi$ for any $k \in \mathbb{Z}$, then the following equalities hold

    \[
    D_N(x) = \frac{\sin(N + \frac{1}{2})x}{2\sin(\frac{x}{2})}
    \]
    \[
    K_N(x) = \frac{2}{N+1}(\frac{\sin(\frac{N + 1}{2})x}{2\sin(\frac{x}{2})})^2
    \]
\end{theorem}
\begin{proof}[Proof idea] We try to express  $D_N(x)\sin(\frac{x}{2}$ in a different form. Here we remember (after a while of course) the forgotten high school formula,
\[
\cos(\alpha)\sin(\beta) = \sin(\alpha + \beta) - \sin(\alpha - \beta)
\]
    Then, the result is a telescoping sum yielding the desired expression.
\end{proof}



     \begin{remark}
         Prove that if \(f\) is integrable, then
    \[
    S_N(f)(x) = \frac{1}{\pi}\int_{-\pi}^{\pi}f(t)D_N(x-t)\,dt
    \]
     \end{remark}
    \begin{proof}
        we start by looking at an arbitrary term of $S_Nf(x)$,

        \begin{align*}
            a_k\cos(kx) + b_k\sin(kx) &= \frac{1}{\pi}\int_{-\pi}^{\pi}\cos(kt)f(t)dt \cdot \cos(kx) + \frac{1}{\pi}\int_{-\pi}^{\pi}\sin(kt)f(t)dt \cdot \sin(kx)\\
        &= \frac{1}{\pi}\int_{-\pi}^{\pi}f(t)(\cos(kt)\cos(kx) + \sin(kt)\sin(kx))dt\\
        &= \frac{1}{\pi}\int_{-\pi}^{\pi}f(t)\cos((x-t)k)dt\\
        \end{align*}
        Now, we add these terms up,
        \begin{align*}
            S_N(x) &=\frac{a_0}{2} + \sum_{k=1}^{N} (a_k\cos(kx) + b_k\sin(kx))\\
            &= \frac{a_0}{2} + \sum_{k=1}^{N}(\frac{1}{\pi}\int_{-\pi}^{\pi}f(t)\cos((x-t)k)dt) \\
            &= \frac{1}{\pi}\int_{-\pi}^{\pi}f(t) (\frac{a_0}{2} + \sum_{k=1}^{N}\cos((x-t)k)dt)) \\
            &= \frac{1}{\pi}\int_{-\pi}^{\pi}f(t)D_N(x-t)dt \qedhere
        \end{align*}

    \end{proof}

     \begin{example}
     Let \(f\) be a function defined as
    \[
    f(x) = \begin{cases}
        \frac{x}{|x|} & x \neq 0\\
        0             & x = 0
    \end{cases}
    \]
    \textbf{(a)} Compute the Fourier coefficients of $f$.

    \textbf{(b)} Prove that
    \[
    (S_{2N} f)(x) = \frac{2}{\pi} \int_{-\pi}^{\pi} \frac{\sin(2Nt)}{\sin(t)} \,dt
    \]
    for $x \in [-\pi,\pi]$ and $N \in \mathbb{N}$.

    \textbf{(c) [Gibbs's phenomenon].} Prove that
    \[
    \lim_{N \to \infty}(S_{2N} f)(\frac{\pi}{2N}) = -\frac{2}{\pi}\int_{-\pi}^{\pi} \frac{\sin(t)}{t} dt \approx 1.179
    \]
    as $N \to \infty$.
     \end{example}


    \begin{proof}[Hint]
        \textbf{(c)} Use uniform convergence to interchange the limit and integral.
    \end{proof}

\section{Summability of Fourier Series}
Continuous functions are known to be "well-behaving" functions in many contexts for many mathematicians. But when it comes to Fourier analysis, the basic (and basis) "Convergence question" becomes very difficult to answer even for continuous functions, of course better than not-continuous ones. Here we replace "convergence of functions" with a simpler kind of "convergence", being summable, and show that Fourier series of any continuous periodic function \(f\) is uniformly summable to \(f\).

\begin{definition}
    A series $\sum_{k=0}^{\infty} a_k$ with partial sums $S_N$ is said to be \textit{Cesaro summable} to $L$ if \textit{Cesaro means}
    \[
    \sigma_N \coloneqq   \frac{S_0 + S_1 + \dots + S_N }{N+1}
    \]
    converges to $L$ as $N \longrightarrow \infty$.
\end{definition}
Following remarks shows that summability is a generalisation of convergence.
\begin{remark}\label{convergence to the same value to the summable limit}
    If $\sum_{k=0}^{\infty} a_k$ converges to $L < \infty$ then it is Cesaro summable to $L$.
\end{remark}
Note that the converse of the remark is not true, as an example take $\sum_{k = 0}^{\infty} (-1) ^ k$.\par
Now, we weaken our \textit{Convergence Question}: \\
\textbf{Summability Question} Given a function $f: \mathbb{R} \xrightarrow{} \mathbb{R} $ periodic on $\mathbb{R}$ and integrable on $[-\pi, \pi]$, is $Sf$ Cesaro summable to \(f\)?\par
We give here a lemma which will make $\sigma_N f$ easy to deal with (especially when we try to estimate $|\sigma_N f - f|$).

\begin{lemma}
    Let $f: \mathbb{R} \xrightarrow{} \mathbb{R} $ periodic on $\mathbb{R}$ and integrable on $[-\pi, \pi]$. Then,
    \[
    \sigma_Nf(x) = \frac{1}{\pi} \int_{-\pi}^{\pi} K_N(t)f(x-t)dt
    \]
    for all $N \in \mathbb{N}$ and for all $ x \in \mathbb{R}$.\footnotemark{}
    \footnotetext{This may remind you the convolution and you are right!}
\end{lemma}

\begin{proof} As standart, we begin with the definition and equal formulations of the expression:
    \begin{align*}
    \sigma_Nf(x) &= \frac{S_0f + S_1f + \dots + S_Nf}{N+1}\\
                 &= \frac{\sum_{k = 0}^{N}S_kf}{N+1} \\
                 &= \frac{\sum_{k = 0}^{N}\frac{1}{\pi}\int_{-\pi}^{\pi}D_k(t)f(x-t)dt}{N+1} &&\text{(see exercise 1.1)}\\
                 &= \frac{1}{\pi}\int_{-\pi}^{\pi}f(x-t)\frac{\sum_{k = 0}^{N}D_k}{N+1}dt &&\text{(interchange sum \& integral for finite series)}\\
                 &= \frac{1}{\pi}\int_{-\pi}^{\pi}f(x-t)K_N(t)dt &&\text{(see the remark for $D_N$ and $K_N$)}\qedheres
    \end{align*}
\end{proof}
\par
To go further, we need to investigate \textit{Fejer kernel}s more. As will be seen, they act similar to the so-called \textit{Dirac-delta} ($\delta$) function.
\begin{lemma}
    For each natural number $N$,
    \begin{align}
        K_N(t) &\geq 0 \text{ for all } t \in \mathbb{R},\\
        \intertext{and}
        &\frac{1}{\pi}\int_{-\pi}^{\pi} K_N(t)dt = 1,\\
        \intertext{for each  $\pi > \delta > 0$,}
        &\lim_{N \to \infty} \int_{\delta}^{\pi}|K_N(t)|dt = 0.
    \end{align}
\end{lemma}
\begin{proof}{Proof idea.}
    For the first one, use Theorem 2 for $x \neq 2\pi k$. For second, use the definition of $K_N$ and for the third, observe that $\sin(\frac{t}{2}) \geq \sin(\frac{\delta}{2})$ for all $t > \delta$ and use Theorem 2.
\end{proof}

Now comes the theorem we have been giving lemma after lemma for just to be able to prove more easily, \textit{Fejer Theorem};

\begin{theorem}[Fejer]
    Suppose that $f: \mathbb{R} \to \mathbb{R}$ is periodic and integrable on each compact interval.

    \footnotemark{*?*} \footnotetext{to be asked!}

        1. If \begin{equation}
             L = \lim_{h \to 0} \frac{f(x_0 + h) + f(x_0 - h)}{2}
        \end{equation}

        exists for some $x_0 \in \mathbb{R}$, then $\sigma_Nf(x_0) \to L$ as $N \to \infty$.   \\
        2. If \(f\) is continuous in some closed interval $I$, then $\sigma_Nf \to f$ uniformly on $I$.

\end{theorem}
Before giving the actual proof, I want to give a "wrong path" in the right direction.
\begin{proof}[Proof trial]
    {As we mentioned above, $K_N$ reminds us $dirac-\delta$ function.}\\
   { We know that,}
    \[
    \int_{-\pi}^{\pi}f(x)\delta(x - x_0)dx = f(x_0) \text{  ,for continuous f.}
    \]
    For non-continuous \(f\), we wonder if we can use this idea for $Fejer kernels$, i.e.
    \[
    \frac{1}{\pi}\int_{-\pi}^{\pi} K_N(t)f(x_0 - t)dt \approx(?) L
    \]\footnotemark{}
    \footnotetext{Unfortunately I could not proceed into this approach, see remark "Summability Kernels" below.}
\end{proof}
Now, we can dive into the actual proof.
\begin{proof}
We start by trying to approximate (as we generally do to show convergence) $|\sigma_N(x_0) - f(x_0)|$. We will, of course, use the lemmas above (otherwise, why give them?). Here we go:
\begin{align*}
|\sigma_N(x_0) - f(x_0)| &= \left|\frac{1}{\pi} \int_{-\pi}^{\pi} K_N(t)f(x_0-t)dt - f(x_0)\right| \quad\text{(lemma 1)} \\
&= \left|\frac{1}{\pi} \int_{-\pi}^{\pi} K_N(t)f(x_0-t)dt - \frac{1}{\pi} \int_{-\pi}^{\pi} K_N(t)f(x_0)dt\right| \quad\text{(lemma 2)} \\
&= \left|\frac{1}{\pi} \int_{-\pi}^{\pi} K_N(t)(f(x_0-t) - f(x_0))dt \right|\\
&= \left|\frac{1}{\pi} \int_{-\pi}^{\pi} K_N(u)(f(x_0+u) - f(x_0))du \right| \quad\text{(u = -t change and evenness of $K_N$)}\\
\end{align*}
Now we add the last two (equal) expressions and divide by 2 to get a term in the form $f(x_0 + h) + f(x_0 - h)$.
\begin{align*}
    &= \left|\frac{1}{2\pi} \int_{-\pi}^{\pi} K_N(u)(f(x_0+u) + f(x_0 - u) - 2f(x_0))du \right| \\
    &= \left|\frac{2}{\pi} \int_{0}^{\pi} K_N(u)(\frac{f(x_0+u) + f(x_0 - u)}{2} - f(x_0))du \right| \quad\text{(evenness of the expression inside)}\\
\end{align*}
Now, let us call $\frac{f(x_0+u) + f(x_0 - u)}{2} - f(x_0)$ as $F(x_0,u)$ for simplicity. since \(f\) is (Darboux) integrable, it is bounded and $M \vcentcolon = \sup_{u \in \mathbb{R}} F(x_0,u)$ exists.
\[
\frac{2}{\pi} \int_{\delta}^{\pi} K_N(u)F(x_0,u)du \leq \frac{2M}{\pi} \int_{\delta}^{\pi} K_N(u)du
\]
which is smaller than $\epsilon$ for large N (see lemma 2). If we have chosen a small $\delta$, $|F(x_0, u)| \leq \epsilon$ since $L$ exists.
\[
\frac{2}{\pi} \int_{0}^{\delta} K_N(u)F(x_0,u)du \leq \frac{2\epsilon}{\pi} \int_{0}^{\delta} K_N(u)du
\]
Last integral approaches to 1 as N goes to infinity (lemma 2). Hence, we managed to show that
\[
|\sigma_N(x_0) - f(x_0)| \leq \epsilon . C \quad\text{(a constant)}
\]
Our choice of $N$s do not depend on $x_0$, hence same arguments show that $\sigma_N \to f$ uniformly (we keep in mind that $I$ is compact, making \(f\) bounded.
\end{proof}

\begin{corollary}{Uniqueness. \footnotemark{} \footnotetext{In Wade, it is named as "Completeness" but Prof. Gheondea preffered this name}} If $f : \mathbb{R} \to \mathbb{R}$ is continuous and periodic, and $a_k(\mathbb{f}) = b_k(\mathbb{f}) = 0$ for $k \in \mathbb{Z}$, then $f(x) = 0$ for all $x \in \mathbb{R}$.
\end{corollary}
\begin{proof}
    Here $\sigma_Nf(x) = 0$ for all real $x$. By the theorem, it approaches uniformly to \(f\) which makes $f \equiv 0$.
\end{proof}
\begin{corollary}
    Let $f : \mathbb{R} \rightarrow \mathbb{R}$ be continuous and periodic. Then there is a sequence of trigonometric polynomials $T_1, T_2, \ldots$ such that $T_N \rightarrow f$ uniformly on $\mathbb{R}$.

\end{corollary}
\begin{proof}
    Set $T_N$ as $\sigma_N$.
\end{proof}
This last corollary can be used to prove a very strange theorem in analysis, \textit{Weierstrass Approximation Theorem}.

\begin{theorem}[Weierstrass Approximation Theorem] Let $[a, b]$ be a closed bounded interval, and suppose that $f : [a, b] \rightarrow \mathbb{R}$ is continuous on $[a, b]$. Given $c > 0$, there exists a (classical) polynomial $P$ on $\mathbb{R}$ such that $|f(x) - P(x)| < c$ for all $x \in [a, b]$.
\end{theorem}

\begin{proof} WLOG take $-a = b = \pi$\\
    We already learned that given a positive $\epsilon$ there exists a trigonometric polynomial $T_{N_\epsilon}$ such that
    \[
    |f - T_{N_\epsilon}|(x) < \epsilon
    \]
    Both $\sin(kx)$ and $\cos(kx)$ are analytic functions \footnotemark{}
    \footnotetext{Meaning they can be written as a power series}. Hence, there exists a classical polynomial $P$ such that
    \[
    |T_{N_\epsilon} - P|(x) < \epsilon
    \]
    By combining two inequalities (and applying necessary algebra) we get
    \[
    |f - P|(x) < \epsilon
    \]
\end{proof}
This theorem shows that (classical and trigonometric) polynomials are dense in the function space $C^{0}[-\infty, \infty]$


\begin{remark}
    Let $E \subset \mathbb{R}$, and suppose that $f, f_k : \mathbb{R} \to \mathbb{R}$ are bounded functions. If $\sum_{k = 0}^{ \infty} f_k$ converges uniformly to $f$ on $E$, then

\[\sigma_N(x) := \sum_{k=0}^{N} (1 - \frac{k}{N+1}) f_k(x)\]

converges to $f(x)$ uniformly on $E$ as $N \to \infty$.

\end{remark}
\begin{proof}{Idea.} Let $S_k$ be the partial sum of the series  $\sum_{t = 0}^{\infty}f_t$. Then,
\[
\sigma_N(x) = \frac{\sum_{k = 0}^{\infty}S_k}{N+1}
\]
We know that for large $N$, $S_k$ is in the $\epsilon$-Neighborhood of \(f\).

\end{proof}
\begin{remark}
    Let
    \[
    S = \frac{a_0}{2} +\sum_{k=1}^{\infty}a_k\cos(kx) + b_k\sin(kx)
    \]
    be a trigonometric series and set
    \[
    \sigma_N(x) = \frac{a_0}{2} +\sum_{k=1}^{\infty}(1 - \frac{k}{N+1})a_k\cos(kx) + b_k\sin(kx)
    \]
    $S$ is the fourier series of $f$, a continuous and periodic function, iff $\sigma_N \to f$ uniformly.
\end{remark}
\begin{proof}
    If $ S = Sf$, then by then by \textit{Fejer theorem}, $\sigma_N = \sigma_Nf \to f$ uniformly.
    Conversely, assume $\sigma_N \to f$ uniformly. Remember that $Sf$ is determined completely by the coefficients. We prove that $a_k = a_k(f)$. Rest is similar.

    \begin{align*}
        a_k(f) &= \frac{1}{\pi} \int_{-\pi}^{\pi} f(x) \cos(kx) dx\\
        &= \frac{1}{\pi} \int_{-\pi}^{\pi} \lim_{N \to \infty}\sigma_N(x) \cos(kx) dx\\
        &= \lim_{N \to \infty} \frac{1}{\pi} \int_{-\pi}^{\pi} \sigma_N(x) \cos(kx) dx \quad \text{(uniform convergence)}\\
        &= \lim_{N \to \infty} (1 - \frac{k}{N+1})a_k \\
        &= a_k
    \end{align*}

\end{proof}
\begin{example}
    Show that
    \[
    S = \sin(\sqrt{2}\pi) + \sum_{k=0}^{\infty}\frac{4(-1)^k\sin(\sqrt{2}\pi)}{2-k^2}\cos(kx)
    \]
    converges uniformly to $\sqrt{2}\pi\cos(\sqrt{2}\pi)$.
\end{example}
\begin{proof}
    Since this problem is given in \textit{Fourier Series}, we look for solutions in that context. First, let us remind that \textit{"If $\sigma_Nf$ converges to $L$ and if we know $S_Nf$ converges, we can conclude $S_Nf$ converges to $L$."}

    \par Without previous remark, it is a difficult series to deal with, but we are lucky. In light of this, we look for a \textit{continuous $f$}. The series $S$ converges (this can be shown by elementary calculus methods), hence if we can find such an \(f\), we are done with the help of the remind and remark above plus "Fejer theorem". \par

    We are given a hint, \(f\) is probably $\sqrt{2}\pi\cos(\sqrt{2}\pi)$. By that choice, $b_k(f) = 0$ and
    \begin{align*}
        a_k(f) &= \frac{1}{\pi} \int_{-\pi}^{\pi} f(x) \cos(kx) dx \\
               &= \sqrt{2} \int_{-\pi}^{\pi}  \cos(\sqrt{2}\pi) \cos(kx) dx \\
               &= \frac{4(-1)^k\sin(\sqrt{2}\pi)}{2-k^2} \quad \text{(left as an exercise!)}
    \end{align*}
\end{proof}

\begin{remark}[Summability Kernels] Let $\phi_N$ be a sequence of real valued, continuous and periodic functions that satisfy
\begin{align*}
    \int_{0}^{2\pi} \phi_N = 1 \text{ and } \int_{0}^{2\pi} |\phi_N| < M < \infty \\
    \intertext{for all $N \in \mathbb{N}$ and }\\
    \lim_{N \to \infty} \int_{\delta}^{2\pi - \delta} |\phi_N| = 0\\
    \intertext{for all $2\pi > \delta > 0$. Suppose that $f$ is continuous and periodic on $\mathbb{R}$}\\
    \lim_{N \to \infty} \int_{0}^{2\pi} f(x - t)\phi_N(t)dt = f(x)\\
    \intertext{uniformly for all $x$.}
\end{align*}
\end{remark}

\section{Growth of Fourier Coefficients}

In previous section, we have seen that (in the light of Fejer Theorem) a continuous function is completely determined by its Fourier coefficients\footnotemark{}\footnotetext{Coefficients determine $\sigma_Nf$ and it converges to $f$}. In this section, we investigate the behaviour of these coefficients, specifically how fast they approach to $0$. \paragraph{}
    Our deep aim is to determine when $Sf$ converges to $f$. In that case, coefficients converge to 0. Yet, we are not able to answer the general convergence question, but we will instead catch a hope, called \textbf{Riemann-Lebesque Lemma}, stating $a_k(f)$ and $b_k(f)$ have to converge to zero. \paragraph{}
    To prove this lemma we use \textbf{Bessel's inequality}.

\begin{theorem}[Bessel's Inequality]

Let $f(x)$ be a function defined on the interval $[-\pi, \pi]$. If $f(x)$ is integrable, the series $\sum_{k=1}{\infty}|a_k(f)|^2 + |b_k(f)|^2$ converges. Moreover, the inequality below holds:
\begin{equation}
    \frac{1}{\pi} \int_{-\pi}^{\pi} |f(x)|^2 \, dx \geq \frac{a_0^2}{2} + \sum_{n=1}^{\infty} \left(a_n^2 + b_n^2\right).
\end{equation}
\end{theorem}
 To prove this theorem, we need a lemma.
 \begin{lemma}
     If $f: \mathbb{R} \to \mathbb{R}$ is integrable on $[-\pi, \pi]$, then,
     \begin{equation}
     \begin{aligned}
         \frac{1}{\pi} \int_{-\pi}^{\pi} f(x)S_Nf(x) \, dx
         &= \frac{a_0^2}{2} + \sum_{n=1}^{N} \left(a_n^2 +
         b_n^2\right)\\
         &=  \frac{1}{\pi} \int_{-\pi}^{\pi} |S_Nf(x)|^2  dx
     \end{aligned}
     \end{equation}
 \end{lemma}
 \begin{proof}[Proof of lemma]
 First equality is coming from the definition of Fourier coefficients, the second one is the job of orthogonalities we gave at the very beginning.
 \[
 \begin{aligned}
     \frac{1}{\pi} \int_{-\pi}^{\pi} f(x)S_Nf(x) \, dx
         &= \frac{1}{\pi} \int_{-\pi}^{\pi} (\frac{a_0(f)f}{2} + \sum_{n=1}^{N} \left(a_n(f)f +
         b_n(f)f\right))dx \\
         &= \frac{1}{\pi} \int_{-\pi}^{\pi}\frac{a_0(f)f}{2} + \sum_{n=1}^{N} \frac{1}{\pi} \int_{-\pi}^{\pi} \left(a_n(f)f +
         b_n(f)f\right))dx \\
         &= \frac{a_0^2}{2} + \sum_{n=1}^{N} \left(a_n^2 +
         b_n^2\right)\\
 \end{aligned}
 \]
 Second equality follows from similar expansions and using orthogonalities.

 \end{proof}
    Now we can prove the theorem. Altough proving the inequality also proves the convergence, it is a genious proof. We shall give a more "natural" proof for convergence.
    \begin{proof}
        Let us first prove the convergence. Let $M$ be the $\frac{\sup(f)}{\pi}$
        \[
        \begin{aligned}
            (\frac{a_0(f)f}{2} + \sum_{n=1}^{N} \left(a_n(f)f +
         b_n(f)f\right)) \\
         &=  \frac{1}{\pi} \int_{-\pi}^{\pi} f(x)S_Nf(x) \, dx \\
         &\leq M \int_{-\pi}^{\pi}S_Nf(x) \, dx \\
         &\leq M \int_{-\pi}^{\pi}(\frac{1}{\pi} \int_{-\pi}^{\pi}f(x-t)D_N(t)dt) \, dx \\
         &\leq M^2  \int_{-\pi}^{\pi}(\frac{1}{\pi} \int_{-\pi}^{\pi}D_N(t)dt) \, dx \\
         &= M^2  \int_{-\pi}^{\pi}(1) \, dx \\
         &= 2\pi M^2 \in \mathbb{R}
        \end{aligned}
        \]
        Hence, for each $N$, we found an upper bound. Since the terms are nn-negative, series converges.

        Now we prove the inequality.
        We fix an $N$ and use the lemma above,
        \[
        \begin{aligned}
            0
            &\leq \frac{1}{\pi} \int_{-\pi}^{\pi} |f(x)-S_Nf(x)| \, dx \\
            &= \frac{1}{\pi} \int_{-\pi}^{\pi}|f|^2 - \left( \frac{a_0^2}{2} + \sum_{n=1}^{N} \left(a_n^2 + b_n^2\right) \right) \quad \text{three equivalences in the lemma}
        \end{aligned}
        \]
        Since $N$ is arbitrary, we can take the limit and we are done.
    \end{proof}

\begin{corollary}[Riemann-Lebesque Lemma]
    \[
    \lim_{N \to \infty}a_k(f) = \lim_{N \to \infty}b_k(f) = 0.
    \]
\end{corollary}

Our next major result is the \textit{Parseval's identity}, when $f$ is continuous and periodic the \textit{Bessel's inequality} actually becomes an equality. To prove that important theorem, we first give a lemma stating $S_Nf$s are the best approximation for $f$ in the following unusual sense\footnotemark{}.
\footnotetext{In measure theory, this lemma tells they are equal in the function space $L^2$}

\begin{lemma}
    Given any trigonometric polynomial $T_N$,
    \[
    \frac{c_0}{2} + \sum_{n=1}^{N} (c_n + d_n)
    \]
    The following inequality holds where $f$ is periodic and integrable:
    \[
    \frac{1}{\pi} \int_{-\pi}^{\pi}\left| f - S_Nf\right|^2 \leq \frac{1}{\pi} \int_{-\pi}^{\pi}\left| f - T_N\right|^2
    \]
\end{lemma}
\begin{proof}
    We will use (6).
    \[
    \begin{aligned}
        \int_{-\pi}^{\pi}\left| f - T_N\right|^2
        &= \int_{-\pi}^{\pi}\left| f - S_Nf + S_Nf - T_N\right|^2\\
        &=  \int_{-\pi}^{\pi} \left| f - S_Nf\right|^2
        + 2 \int_{-\pi}^{\pi}  (f - S_Nf)(S_Nf - T_N)
        + \int_{-\pi}^{\pi} \left| T_N - S_Nf\right|^2 \\
        &\geq \int_{-\pi}^{\pi} \left| f - S_Nf\right|^2
        + 2 \int_{-\pi}^{\pi}  (f - S_Nf)(S_Nf - T_N)  \quad(\ast)
    \end{aligned}
    \]
    We claim that the last term is zero. Indeed,
    \[
    \begin{aligned}
        \frac{1}{\pi}\int_{-\pi}^{\pi}  (f - S_Nf)(S_Nf - T_N)
        &= \frac{1}{\pi} \int_{-\pi}^{\pi}  (fS_Nf - S_Nf^2) + \frac{1}{\pi} \int_{-\pi}^{\pi} (-fT_N + S_NfT_N) \quad \text{(first term is zero, (6))}\\
        &= \frac{1}{\pi} \int_{-\pi}^{\pi} -fT_N + \frac{1}{\pi} \int_{-\pi}^{\pi}  S_NfT_N\\
        &= -\left( \frac{a_0(f)c_0}{2} + \sum_{j=1}{N}a_j(f)c_j + b_j(f)d_j\right) \\
        &+ \left( \frac{a_0(f)c_0}{2} + \sum_{j=1}{N}a_j(f)c_j + b_j(f)d_j\right) = 0
    \end{aligned}
    \]
    Hence, by $(\ast)$ proof is done.
\end{proof}
We can now proceed to our next major result, a famous identity by Parseval.

\begin{theorem}[Parseval's identity]\footnotemark{}
    If $f$ is continuous and periodic, then,
        \begin{equation}
            \frac{a_0^2}{2} + \sum_{n=1}^{\infty} \left(a_n^2 + b_n^2\right) = \frac{1}{\pi} \int_{-\pi}^{\pi} |f(x)|^2 \, dx.
        \end{equation}

\end{theorem}
\footnotetext{This theorem is used to prove one of the most beautiful formulas in mathematics, namely \textit{Bassel problem}. If we set $f(x) = x$ (and restrict it to be periodic) then we get the famous equality: $\sum_{k=1}^{\infty}\frac{1}{k^2} = \frac{\pi^2}{6}$.}
\begin{proof}
    We use \textit{Fejer theorem} and interchangibility of limit and integral in the case of uniform convergence.
    \[
    \begin{aligned}
        \frac{1}{\pi} \int_{-\pi}^{\pi} |f(x)|^2 \, dx
        &= \frac{1}{\pi} \int_{-\pi}^{\pi} f\lim_{N \to \infty}\sigma_N \, dx\\
        &= \lim_{N \to \infty}\frac{1}{\pi} \int_{-\pi}^{\pi} f\sigma_N \, dx\\
        &= \lim_{N \to \infty}\frac{1}{\pi} \int_{-\pi}^{\pi} f\left( \sum_{k=0}^{N} (1 - \frac{k}{N+1}) a_k(f)\cos(kx) + b_k(f)\sin(kx) \right)\, dx\\
        &= \lim_{N \to \infty}\frac{1}{\pi} \int_{-\pi}^{\pi} \left( \sum_{k=0}^{N} (1 - \frac{k}{N+1}) fa_k(f)\cos(kx) + fb_k(f)\sin(kx) \right)\, dx\\
        &= \lim_{N \to \infty}  \sum_{k=0}^{N} \left( \frac{1}{\pi} \int_{-\pi}^{\pi}(1 - \frac{k}{N+1}) fa_k(f)\cos(kx) + fb_k(f)\sin(kx)\, dx \right)\\
        &= \lim_{N \to \infty}  \sum_{k=0}^{N} \left((1 - \frac{k}{N+1}) a_k(f)^2 + b_k(f)^2 \right)\\
        &=   \sum_{k=0}^{N} \lim_{N \to \infty}\left((1 - \frac{k}{N+1}) a_k(f)^2 + b_k(f)^2 \right)\\\
        &=   \sum_{k=0}^{N}  a_k(f)^2 + b_k(f)^2 \\
    \end{aligned}
    \]
\end{proof}

The Riemann-Lebesgue Lemma can be improved if f is smooth and periodic. In
fact, the following result shows that the smoother f is, the more rapidly its Fourier
coefficients converge to zero.

\begin{theorem}\label{growth of coefficients in C^j}

    Let $f$ be a real valued function and $j \in \mathbb{N}$. If $f^{(j)}$ exists and integrable on $[-\pi,\pi]$ and $f^{(l)}$ is periodic for each $0 \leq l < j$, then
    \begin{equation}
        \lim_{k \to \infty} k^j a_k(f) = \lim_{k \to \infty} k^j b_k(f) = 0
    \end{equation}

\end{theorem}

This theorem does not only tell us how fast the coefficients but also gives an important result for the convergence of \textit{second degree good} (class $C^2$-periodic) functions. But as we are in a mathematics document, let's give first the proof.

\begin{proof}[Proof idea.]
    Our strategy is to find make the expression $k^ja_k(f)$ a Fourier coefficient of another function and then apply the \emph{Riemann-Lebesque lemma}. We find that the required function is $f^{(j)}$ and rest of the proof follows from integration by parts and $f^{(l)}(-\pi)\cos(-k\pi)$ being equal to $f^{(l)}(\pi)\cos(k\pi)$ by periodicity.
\end{proof}

Now, we will use this theorem to prove a convergence fact.

\begin{corollary}\label{C^2 converges}
    If $f: \RR \to \RR$ belongs to $\cC^2(\RR) $ and $f, f^{'}$ are periodic, then $Sf$ converges to $f$ uniformly and absolutely on $\RR$.
\end{corollary}

\begin{proof}
    By the theorem, for sufficiently large $n$, $a_n(f), b_n(f) < \frac{1}{n^2}$. Hence, $S_N(f) <  \frac{a_0}{2} + \sum_{n=1}^{\infty} \left(a_n + b_n\right) < \infty$. Therefore the series converge absolutely. For the uniform convergence, we use \textit{Fejer theorem} and convergence of $S_Nf$ to the same value with $\sigma_N f$ \footnotemark{}\footnotetext{check again!}
\end{proof}

We want to make the assumptions of theorems as less as possible. In this quest, a new class of \emph{well-behaving} functions should be defined: \emph{functions of bounded variations.}

\footnotemark{*?*} \footnotetext{to be asked!}

After defining and some investigation on this new class, we will give a theorem on the growth of the coefficients of these kinds of functions.



\subsection{Functions of bounded variations}

In this part, we will deal with the functions that do not wiggle too much \footnotemark{}
\footnotetext{A classic example of wiggling too much function is $\sin(\frac{1}{x})$ near 0.} which are defined while dealing with our \emph{Convergence question.}
\begin{definition}
    Let $\phi : [a,b] \to \RR$ and $P = \{x_0, \dots, x_n\} \in IP[a,b]$.
\footnotemark{}
\footnotetext{Interval partitions of $[a,b]$}
Set
\[
V(\phi,P) = \sum_{k=0}^{n-1}|\phi(x_{k+1} - \phi(x_{k}|
\]
The \emph{variation} of $\phi$ is defined as

\begin{equation}
    Var(\phi) \coloneqq \sup\{V(\phi,P)| P \in IP[a,b]\}
\end{equation}
$\phi$ is said to be \emph{of bounded variation} if $Var(\phi) < \infty$. \footnotemark{}\footnotetext{This definition is the analogue of \textit{rectifiable} in \textit{Vector analysis}.}
\end{definition}

We will give some facts on this class of functions without proofs. For details, check \cite{Wade}.

\begin{remark} \label{bounded var. implications}The following implications hold true for functions of bounded variations \emph{only} in the given directions:
    \begin{itemize}
        \item[(i)] If  $f \in \cC^1[a,b]$, then $f$ is of bounded variation on $[a, b]$.
        \item[(ii)] If  $f$ is monotone, then it is of bounded variation on $[a, b]$.
        \item[(iii)] If $f$ is of bounded variation, then it is bounded
    \end{itemize}
\end{remark}

\begin{remark}
     $f$ is of bounded variation on $[a,b]$ iff there exists two monotone functions $h,g$ such that $f = h - g$.
\end{remark}

\begin{corollary}\label{facts on bounded var.}
    Because of monotonicity, following assertions are true when $f$ is of bounded variations:
    \begin{itemize}
        \item[(a)] For each \(x \in [a, b)\), \(f(x+)\) exists, and for each \(x \in (a, b]\), \(f(x-)\) exists.
        \item[(b)] $f$ has no more than countably many points of discontinuity in $[a,b]$.
        \item[(c)] $f$ is integrable on $[a,b]$.
    \end{itemize}
\end{corollary}

Now, we return to our original topic, \emph{Fouier series}. We saw that if $f$ is of class $\cC^1$, then $ka_k(f)$ approach to zero (see \ref{growth of coefficients in C^j}). It is natural to ask that what happens in a more general case, functions of bounded variations (for being general, see \ref{bounded var. implications}). Let's first try an informal (or at physicist's standards, formal) calculation to approximate the bound. With the help of Riemann sums, Mean value theorem and $\sin(kx_0) = \sin(kx_n) = 0$, we can make the following approximation:
\[
\begin{aligned}
    \pi a_k(f) = \int_{-\pi}^{\pi}f(x) \cos(kx) dx &\approx \sum_{i = 1}^{n}f(x_i)\cos(kx_i)(x_{i+1} - x_i)\\
    &\approx \frac{1}{k} \sum_{i = 1}^{n}f(x_i)(\sin(kx_{i+1} - \sin(kx_i))\\
    &= \frac{1}{k} \sum_{i = 1}^{n}(f(x_{i+1}) - f(x_i)) \sin(kx_i)\\
\end{aligned}
\]
Since the absolute value of the last sum is bounded by $Var(f)$, we make the claim $|ka_k(f)| < \frac{Var(f)}{\pi}$. This claim can easily be proven with integration by parts, Fundamental Theorem and periodicity if $f$ is increasing, periodic and differentiable. But we do not stop here and prove for more general case:

\begin{lemma}\label{bounded var. coefficient grow}
    Suppose that $f,\phi$ are periodic where $f$ is of bounded variation and $\phi$ is continuously differentiable. If $M \coloneqq \sup{|\phi|}$, then
    \begin{equation}
        \left|\int_{-\pi}^{\pi}f(x)\phi'(x)dx\right| \leq M Var(f)
    \end{equation}
\end{lemma}
\begin{proof}
    Later
\end{proof}

Now we can link this to Fourier series.
\begin{theorem}\label{growth of ceof. of bdd. var.}
    If $f$ is periodic and of bounded variation, then
    \[
    |ka_k(f)|\leq \frac{Var(f)}{\pi}  \text{ and }  |kb_k(f)| \leq \frac{Var(f)}{\pi}
    \]
    $\forall k \in \NN$.
\end{theorem}

Proof immediately follows from \ref{bounded var. coefficient grow} with the choice $\phi = \cos(kx)$. Now, we have concluded our discussion of bounded variation for this section. Before jumping into the next section, we will see some important remarks which are either corollaries of the results above or equivalent versions of them.

\begin{remark}\label{alternative riemann-lebesque lemma}
    If $f$ is integrable on $[-\pi, \pi]$ and $\alpha \in \RR$, then

    \footnotemark{*?*} \footnotetext{What to do when k is real number?, should I state k is natural?}


    \[
    \lim_{k \to \infty}\int_{-\pi}^{\pi}f(x)\cos((k+\alpha)x) dx = 0
    \]
\end{remark}
This remark tells us to use \emph{Riemann - Lebesque lemma}. But we will also give an independent proof to see that this theorem is actually equvalent to the lemma.

\begin{proof}[Proof by \textit{Riemann - Lebesque lemma}]
    Expand $\cos((k+\alpha)x)$ and then apply the lemma.
\end{proof}
\begin{proof}\cite{Measure}\footnotemark{} \footnotetext{The body of the book is taken from \cite{Measure}}
    We will just prove the case $\alpha = 0$ and leave the rest ($\alpha \neq 0$ and the other coefficient $b_k(f)$) as an exercise. Via a change of variables ($x = u + \frac{\pi}{k}$) we get
%
%ask that part
%could not use intertext to make it all in aligned
    \[
    \begin{aligned}
        a_k \coloneqq \int_{-\pi}^{\pi}f(x)\cos(kx) dx &= \int_{-\pi}^{\pi}f(u + \frac{\pi}{k})\cos(ku + \pi) du\\
        &= - \int_{-\pi}^{\pi}f(u + \frac{\pi}{k})\cos(ku) du\\
        \end{aligned}
    \]
        by using the fact $|\cos(kx)| < 1$we get
    \[
        \int_{-\pi}^{\pi}|f(u + \frac{\pi}{k}) - f(u)|du \geq \left|\int_{-\pi}^{\pi}(f(u + \frac{\pi}{k}) - f(u))\cos(ku)du\right| = 2|a_k|
    \]
All we need to do is to show that the first integral converges to zero.\footnotemark{}

 \footnotetext{This is most easily done by \emph{General Weierstrass Approximation Theorem} stating for every (Lebesque) integable function $f$ there exists a continuous function in the $\epsilon$-neighbourhood of $f$.}

From measure theory, we know that if a function is (Riemann) integrable, then it has at most countably many discontinuities. We also know that countably many values do not contribute to integral, therefore what forms the value of the integral is the points where $f$ is continuous at. At these points, $f(x + \frac{\pi}{k}) \to f(x)$ as $k \to \infty$, hence, $|a_k|$ goes to zero.


 \footnotemark{*?*}\footnotetext{to be asked}


\end{proof}

The following remark shows a useful application of Fourier series to approximate the differential of a function.

\begin{remark}
    If $f: \RR \to \RR$ belongs to $\cC^\infty(\RR)$ and $f^{(j)}$ is periodic for all $j \geq 0$, then $Sf$ is term-by-term differentiable on $\RR$. In fact,
\[
\frac{\de ^jf}{{\de x}^j}(x) = \sum_{k = 1}^{\infty} \frac{\de ^j}{{\de x}^j}(a_k(f)\cos(kx) + b_k(f)\sin(kx))
\]
\end{remark}

\begin{proof}
    By (\ref{C^2 converges}), $Sf$ converges absolutely and uniformly. A theorem

    \footnotemark{} \footnotetext{Check the related parts of \cite{Wade} for more information. }

    form the branch \emph{series of function} gives us the necessary argument to claim that equality holds for every bounded $E \subset \RR$.


    \footnotemark{*?*}\footnotetext{to be asked}



\end{proof}

Next remark can be used to give a different proof \emph{Riemann-Lebesque lemma} for continuous functions.

\begin{remark}
    Let $f: \RR \to \RR$ be a continuous and periodic function. The \emph{Modulus of continuity} is defined as
    \[
    \omega(f, \delta) \coloneqq sup_{|h| < \delta}|f(t+h) - f(t)|
    \] The following assertians hold.
    \begin{itemize}
        \item[(i)] $a_k(f) = \frac{1}{2\pi}  \int_{-\pi}^{\pi} (f(u) - f(u + \frac{\pi}{k}))\cos(kx)dx$
        \item[(ii)] $|a_k(f)| \leq \omega(f, \frac{\pi}{k})$ and $|b_k(f)| \leq \omega(f, \frac{\pi}{k})$
    \end{itemize}
\end{remark}

Proof can be done via the technique in (\ref{alternative riemann-lebesque lemma}) and approximations of integrals. Since $f$ is continuous, modulus of convergence converges to zero as $k \to \infty$, hence, the coefficients converge to zero for continuous functions.

\par Now we have necessary tools to deal with the big \emph{Convergence Question!}

\section{Convergence of Fourier Series}
In this section, we will show that under certain conditions a summable series must also be convergent. These kinds of results are called \emph{Tauberian theorems}.

\begin{theorem}[Tauberian]
    Let $a_k \geq 0$ and $L \in \RR$. If series is Ceasaro summable to $L$, then
    \[
    \sum_{k=0}^{\infty} a_k = L
    \]
\end{theorem}

\begin{proof}
    We notice that  $(1-\frac{k}{n})a_k \leq a_k \leq 2(1-\frac{k}{n})a_k$. We add first $n$ terms and get
    \[
    \sigma_n \leq S_n \leq 2 \sigma_2n
    \]
    The rest is the job of \textit{Sandwich theorem}.
\end{proof}
\begin{proof}[Alternative proof from \cite{Wade}]
    It will be sufficient by (\ref{convergence to the same value to the summable limit}) to show that series converges Assume to the contrary that it diverges. Since all terms are non-negative, for every $M > 0$, there exists $n_0 \in \NN$ such that for $n > n_0$, $S_n > M$. Take $ N > n_0$
    \[
    \sigma_N = \frac{S_0 + \dots + S_{n_0}}{N+1} + \frac{S_{n_0 + 1} + \dots + S_{N}}{N+1} \geq 0 + \frac{N-n_0}{N+1}M
    \]
    Taking the limits of both sides yields $L > M$ for every $M > 0$. Contradiction.
\end{proof}


We can use this result to improve \emph{Riemann - Lebesque lemma} for certain functions.

\begin{corollary}
    Let $f$ be periodic on $\RR$ and integrable. If $a_k(f) = 0$ and $b_k(f) \geq 0$, then
    \[
    \sum_{k=1}^{\infty}\frac{b_k(f)}{k} < \infty
    \]
\end{corollary}


\begin{proof}
    The convergent series looks like a Fourier series of a function, say $F$. If we also find that $F$ is continuous, then we can combine \textit{Fejer} and \textit{Tauberian} and get the convergence. The division by $k$ gives us a hint, therefore we guess
    \[
    F(x) = \int_{0}^{x}f(t)dt
    \]
    $F$ is continuous. Since $a_0(f) = 0$,  $F$ is also periodic. Integration by parts yields
    \[
    a_k(F) = \frac{b_k(f)}{k} \geq 0 \text{ and } b_k(F) = 0.
    \]
    The series we investigate is $SF(0)$. By \emph{Fejer}, $\sigma_NF(0) \to F(0) = 0$. By \emph{Tauber}, the series converges.
\end{proof}


Now, we will show that the converse of the \emph{Riemann - Lebesque lemma} is false.


\begin{example}
    Terms of the series
    \[
    \sum_{k=2}^{\infty}\frac{\sin(kx)}{\log(k)}
    \]
    converges to zero. But if it was a Fourier series of some function, then by the above corollary,
    \[
    \sum_{k=2}^{\infty}\frac{1}{k\log(k)}
    \]
    would converge, which is false.
\end{example}


The following theorem is one of the deepest \emph{Tauberian type} results.


\begin{theorem}[Hardy]\label{Hardy}
    Let $E \subset \RR$ and suppose that $f_k: \RR \to \RR$ is a sequence of functions that satisfy
    \[
    |kf_k(x)| < M
    \]
    for all $x \in \RR$, all $k \in \NN$ and some $M > 0$. If $\sum_{k=0}^{\infty}f_k$ is uniformly Cesaro summable to $f$ on $E$,
    then it converges to $f$ uniformly on $E$.
\end{theorem}


\begin{proof}

\footnotemark{*?*}\footnotetext{To be asked! Probably wrong but why?}


    We try to write partial sums in terms of Cesaro sums:
    \[
    S_n = \sigma_n (n+1) - \sigma_{n-1} n
    \]
    For sufficiently large $n$, $\sigma_n$ is in $\epsilon$-neighborhood of $f$.
    \[
  f - \epsilon = (f-\epsilon)(n+1) -  (f+\epsilon)n < S_n = \sigma_n (n+1) - \sigma_{n-1} n < (f+\epsilon)(n+1) -  (f-\epsilon)n =  f + \epsilon
    \]
    Hence, $|S_n - f| < \epsilon$ meaning $S_n$ converges to $f$.
\end{proof}


We are now in a position to answer the big \emph{Convergence Question.}

\begin{theorem}[Dirichlet - Jordan]
    If a real valued function $f$ is periodic of bounded variation on $[-\pi,\pi]$, then
    \[
    \lim_{n \to \infty}S_nf(x) = \frac{f(x+) + f(x-)}{2}
    \]

    If $f$ is also continuous on some closed interval $I$, then
    \[
    \lim_{n \to \infty}S_nf = f
    \]
    uniformly on $I$.
\end{theorem}


\begin{proof}
    By (\ref{facts on bounded var.}), one sided limits exists and $f$ is integrable. Hence, assumptions of \emph{Fejer theorem} are satisfied and Cesaro sums converge the value at the right. By (\ref{growth of ceof. of bdd. var.}) we get
    \[
    |ka_k(f)|\leq \frac{Var(f)}{\pi}  \text{ and }  |kb_k(f)| \leq \frac{Var(f)}{\pi}
    \]
    Hence, assumptions of (\ref{Hardy}) are satisfied. $S_Nf$ converge to the same value with Cesaro sums.
\end{proof}








\begin{thebibliography}{99}

\bibitem{Wade} \textsc{William R. Wade},  \textit{Introduction to analysis}, \textbf(2010).

\bibitem{Measure} \textsc{M. Capinski, E. Kopp}, \textit{Measure, Integral and Probability}, Springer,
 1994.

\end{thebibliography}
\end{document}
