\documentclass{beamer}


\usepackage{comment}
\usepackage{graphicx}
\usepackage{animate}
\usepackage[utf8]{inputenx} % For æ, ø, å
\usepackage{babel}          % Automatic translations
\usepackage{csquotes}       % Quotation marks
\usepackage{microtype}      % Improved typography
\usepackage{amssymb}        % Mathematical symbols
\usepackage{mathtools}      % Mathematical symbols
\usepackage[absolute, overlay]{textpos} % Arbitrary placement
\setlength{\TPHorizModule}{\paperwidth} % Textpos units
\setlength{\TPVertModule}{\paperheight} % Textpos units
\usepackage{tikz}
\usetikzlibrary{overlay-beamer-styles}  % Overlay effects for TikZ




\newtheorem{conjecture}[theorem]{Conjecture}
\newtheorem{remark}[theorem]{Remark}
\newtheorem{remarks}[theorem]{Remarks}
% The following paragraph writes the equation numbers with two counters,
% the first is the section number and the second resets within the section.
\makeatletter
\@addtoreset{equation}{section}
\makeatother
\renewcommand\theequation{\arabic{section}.\arabic{equation}}
% The next paragraph defines macros for special roman letters to be used
\newcommand{\CC}{{\mathbb C}} % the set of complex numbers
\newcommand{\NN}{{\mathbb N}} % the set of natural numbers
\newcommand{\QQ}{{\mathbb Q}} % the set of rational numbers
\newcommand{\ZZ}{{\mathbb Z}} % the set of integer numbers
\newcommand{\DD}{{\mathbb D}} % the unit disk
\newcommand{\RR}{{\mathbb R}} % the set of real numbers
\newcommand{\TT}{{\mathbb T}} % the unit circle (the one dimensional torus)
% The next paragraph defines macros for caligraphic letters
\newcommand{\cA}{{\mathcal A}}
\newcommand{\cB}{{\mathcal B}}
\newcommand{\cC}{{\mathcal C}}
\newcommand{\cD}{{\mathcal D}}
\newcommand{\cE}{{\mathcal E}}
\newcommand{\cF}{{\mathcal F}}
\newcommand{\cG}{{\mathcal G}}
\newcommand{\cH}{{\mathcal H}}
\newcommand{\cJ}{{\mathcal J}}
\newcommand{\cK}{{\mathcal K}}
\newcommand{\cL}{{\mathcal L}}
\newcommand{\cM}{{\mathcal M}}
\newcommand{\cN}{{\mathcal N}}
\newcommand{\cP}{{\mathcal P}}
\newcommand{\cQ}{{\mathcal Q}}
\newcommand{\cR}{{\mathcal R}}
\newcommand{\cS}{{\mathcal S}}
\newcommand{\cT}{{\mathcal T}}
\newcommand{\cU}{{\mathcal U}}
\newcommand{\cV}{{\mathcal V}}
\newcommand{\cW}{{\mathcal W}}
\newcommand{\cZ}{{\mathcal Z}}

\author{A. Eren Uyanık}
\title{Classical Fourier Series}
\subtitle{Mid-semester presentation}
\newcommand{\univ}{Bilkent University}
\newcommand{\advisor}{Aurelian Gheondea}



\setbeamertemplate{title page}{%
  \vspace{2cm}
  \begin{center}
    \usebeamerfont{title}\inserttitle\par
    \usebeamerfont{subtitle}\usebeamercolor[fg]{subtitle}\insertsubtitle\par
    \vspace{1cm}
    \usebeamerfont{author}\insertauthor\par
    \vspace{0.5cm} % Adjust the spacing as needed
    \usebeamerfont{institute}\univ\par % Add the university
    \vspace{0.3cm} % Adjust the spacing as needed
    \usebeamerfont{institute}\advisor\par % Add the advisor
  \end{center}
  \vfill
}




\begin{document}

\begin{frame}[plain]
  \titlepage
\end{frame}






\section{Overview}


% Use
%
%     \begin{frame}[allowframebreaks]{Title}
%
% if the TOC does not fit one frame.
\begin{frame}{Table of contents}
    \tableofcontents[currentsection]
\end{frame}


\section{History and Motivation}



\begin{frame}{History and Motivation}
\small
\begin{enumerate}
    \item Propagation of heat. 
    \item Birth of the definitions of ...
    \begin{itemize}
        \item Functions
        \item Types of convergence (absolute, uniform, pointwise)
        \item Riemann and Lebesque integrals
        \item (Cantor's) Set theory
    \end{itemize}
\end{enumerate}
\end{frame}




\section{Introduction}
%\SectionPage


\begin{frame}{Trigonometric Series}
\small
    \begin{definition}
        A trigonometric series $S$ is defined as 
        \[
        S(x) = \frac{a_0}{2} +  \sum_{k = 1}^{\infty} (a_k  \cos(kx) + b_k  \sin(kx))
        \]
        Partial sums, $S_N$, are called as trigonometric polynomial.
        
    \end{definition}
    \begin{definition}[Orthogonality]
        Two (integrable) functions $f, g$ are called orthogonal if,
        \[
        \int_{-\pi}^{\pi}f.g = 0
        \]
    \end{definition}
        
\end{frame}
\begin{frame}{Some Orthogonalities}
\small
    Before giving the definition of \textit{Fourier Series}, we give some useful identities.

    \[
    \int_{-\pi}^{\pi} \cos(mx)cos(nx) = \begin{cases} 
                                        0 & m \neq n \\
                                      \pi &  m = n \neq 0\\
                                     2\pi & m = n = 0
                                        \end{cases}
    \]
    \[
    \int_{-\pi}^{\pi} \sin(mx)sin(nx) = \begin{cases} 
                                        0 & m \neq n \\
                                      \pi &  m = n \neq 0\\
                                        0 & m = n = 0
                                        \end{cases}
    \]
    \[
    \int_{-\pi}^{\pi} \sin(mx)cos(nx) = 0  
    \]
\end{frame}
\begin{frame}[shrink=9]{Fourier Series}
\small
    \begin{definition}
        Let \(f\) be an integrable function on [$\pi, \pi$]. Then, Fourier series of \(f\), namely $S($\(f\)) is defined as
    \[
    S(f)(x) = \frac{a_0(f)}{2} +  \sum_{k = 1}^{\infty} (a_k(f)  \cos(kx) + b_k(f)  \sin(kx))
    \]
    where the coefficients $a_k(f)$ and $b_k(f)$ are defined as

    \[
    a_k(f) = \frac{2}{\pi}\int_{-\pi}^{\pi} f(x) \cos(kx)\,dx
    \]
    \[
    b_k(f) = \frac{2}{\pi}\int_{-\pi}^{\pi} f(x) \sin(kx)\,dx
    \]
    \end{definition}

    \begin{theorem}[Fourier]
    If a trigonometric series 
    \[
    S := \frac{a_0}{2} +  \sum_{k = 1}^{\infty} (a_k  \cos(kx) + b_k  \sin(kx))
    \]
   converges uniformly to a function \(f\), the $S$ is the Fourier series of \(f\), i.e. $S(f)$.
\end{theorem}
\end{frame}




\section{Summability}




\begin{frame}{Cesaro Sums}
    We ask whether the Fourier Series of $f$ converge to $f$ or not. It turns out to be really difficult question. Here comes a weaker version of convergence.
    \begin{definition}[Cesaro Summable]
        A series $\sum_{k=0}^{\infty} a_k$ with partial sums $S_N$ is said to be \textit{Cesaro summable} to $L$ if \textit{Cesaro means}
    \[
    \sigma_N \vcentcolon =  \frac{S_0 + S_1 + \dots + S_N }{N+1}
    \]
    converges to $L$ as $N \longrightarrow \infty$.
    \end{definition}
\end{frame}




    \begin{frame}{Two Kernels}
    \small
    In general, we will not deal with the original definitions but the equivalent versions of them. Two kernels defined here will make theorems a lot easier to prove via equivalences given as convolutions.
    \begin{definition}[Kernels]
        Let $N$ be a nonnegative integer.
\begin{enumerate}
  \item[(i)] The Dirichlet kernel of order $N$ is the function defined, for each $x \in \mathbb{R}$, by
  \[
  D_N(x) = \begin{cases}
            \frac{1}{2} & \text{if } N = 0, \\
            \frac{1}{2}  + \sum_{k=1}^{N} \cos(kx) & \text{if } N \in \mathbb{N}.
           \end{cases}
  \]
  
  \item[(ii)] The Fejer kernel of order $N$ is the function defined, for each $x \in \mathbb{R}$, by
  \[
  K_N(x) = \begin{cases}
            \frac{1}{2} & \text{if } N = 0, \\
            \frac{1}{2} + \sum_{k=1}^{N}\left( 1 - \frac{k}{N + 1} \right)\cos(kx) & \text{if } N \in \mathbb{N}.
           \end{cases}
  \]
\end{enumerate}
    \end{definition}

\end{frame}



\section{Kernels}

\begin{frame}{Investigation of Kernels}
    \begin{remark}
         If $N$ is a nonnegative integer, then
\[
K_N(x) = \frac{D_0(x) + D_1(x) + \dots + D_N(x)}{N+1}
\]
for all $x \in \mathbb{R}$.
    \end{remark}

    \begin{theorem}
         If $x$ cannot be written in the form $2k\pi$ for some $k \in \mathbb{Z}$, then the following equalities hold

    \[
    D_N(x) = \frac{\sin\left(N + \frac{1}{2}\right)x}{2\sin(\frac{x}{2})} \text{,}
    \]
    \[
    K_N(x) = \frac{2}{N+1}\left(\frac{\sin\left(\frac{N + 1}{2}\right)x}{2\sin(\frac{x}{2})}\right)^2 \text{.}
    \]
    \end{theorem}
\end{frame}




\begin{frame}[shrink = 9]{Equivalent Definitions via the Kernels}
    \begin{theorem} If \(f\) is integrable, then,
        \begin{itemize}
            \item[(i)]
    \[
    S_N(f)(x) = \frac{1}{\pi}\int_{-\pi}^{\pi}f(t)D_N(x-t)\,dt \text{.}
    \]
            \item[(ii)] 
    \[
    \sigma_Nf(x) = \frac{1}{\pi} \int_{-\pi}^{\pi} K_N(t)f(x-t)dt
    \]
        \end{itemize}
        for all $N \in \mathbb{N}$ and for all $ x \in \mathbb{R}$ where $D_N$ is the Dirichlet kernel and $K_N$ is the Fejer kernel.
    \end{theorem}

    \begin{lemma} For each natural number $N$, the following properties hold:
    \begin{align*}
        K_N(t) &\geq 0 \text{ for all } t \in \mathbb{R}, \\
        \frac{1}{\pi}\int_{-\pi}^{\pi} K_N(t) \, dt &= 1, \\
        \lim_{N \to \infty} \int_{\delta}^{\pi} |K_N(t)| \, dt &= 0 \text{ for each $\pi > \delta > 0$.}\\ 
    \end{align*}
    \end{lemma}
\end{frame}





\begin{frame}{Semi-Convergence}
\small
\begin{theorem}[Fejer]
    Suppose that $f: \mathbb{R} \to \mathbb{R}$ is periodic and integrable on each compact interval.
    
        1. If \begin{equation*}
             L = \lim_{h \to 0} \frac{f(x_0 + h) + f(x_0 - h)}{2}
        \end{equation*}
       
        exists for some $x_0 \in \mathbb{R}$, then $\sigma_Nf(x_0) \to L$ as $N \to \infty$.   \\
        2. If \(f\) is continuous in some closed interval $I$, then $\sigma_Nf \to f$ uniformly on $I$.
    
\end{theorem}

An important theorem can be proven using \textit{Fejer Theorem}:
\begin{theorem}[Weierstrass Approximation Theorem]
Let $[a, b]$ be a closed bounded interval, and suppose that $f : [a, b] \rightarrow \mathbb{R}$ is continuous on $[a, b]$. Given $c > 0$, there exists a (classical) polynomial $P$ on $\mathbb{R}$ such that $|f(x) - P(x)| < c$ for all $x \in [a, b]$.
    
\end{theorem}
\end{frame}



\section{Fourier Coefficients}


\begin{frame}{Bassel's Inequality }

    \begin{theorem}[Bessel's Inequality]\label{bassel inequality}
Let $f(x)$ be a function defined on the interval $[-\pi, \pi]$. If $f(x)$ is integrable, the series $\sum_{k=1}^{\infty}a_k(f)^2 + b_k(f)^2$ converges. Moreover, the inequality below holds:
\begin{equation*}
    \frac{1}{\pi} \int_{-\pi}^{\pi} |f(x)|^2 \, dx \geq \frac{a_0(f)^2}{2} + \sum_{n=1}^{\infty} \left(a_n(f)^2 + b_n(f)^2\right).
\end{equation*}
\end{theorem}

\begin{lemma}\label{lemma needed to prove basels inequality}
     If $f: \mathbb{R} \to \mathbb{R}$ is integrable on $[-\pi, \pi]$, then,
     \begin{equation*}
     \begin{aligned}
         \frac{1}{\pi} \int_{-\pi}^{\pi} f(x)S_Nf(x) \, dx
         &= \frac{a_0(f)^2}{2} + \sum_{n=1}^{N} \left(a_n(f)^2 +
         b_n(f)^2\right)\\
         &=  \frac{1}{\pi} \int_{-\pi}^{\pi} |S_Nf(x)|^2  dx
     \end{aligned}
     \end{equation*}
 \end{lemma}
    
\end{frame}




\begin{frame}{Convergence to 0}
    \small
    We know that if a series converges, then its terms converge to zero. Here, we get a reason to hope $Sf$ will converge to $f$.
    \begin{theorem}[Riemann-Lebesque Lemma]
        If $f$ is integrable on $[-\pi,\pi]$, then
        \[
        \lim_{k \to \infty}a_k(f)  = \lim_{k \to \infty}b_k(f) = 0
        \]
    \end{theorem}

    \begin{theorem}[Parseval's Identity]
        If $f: \mathbb{R} \to \mathbb{R}$ is periodic and continuous, then
        \[
        \frac{a_0(f)^2}{2} + \sum_{k = 1}^{\infty}(a_k(f)^2 + b_k(f)^2) = \frac{1}{\pi}\int_{-\pi}^{\pi}f^2
        \]
    \end{theorem}
\end{frame}




\begin{frame}{Speed of convergence to 0}
    Depending of the properties of a given function, we can estimate how fast the coefficients will converge to 0.
    \begin{theorem}
        Given a function \( f : \mathbb{R} \to \mathbb{R} \) and \( j \in \mathbb{N} \). If \( f^{(j)} \) exists and is integrable on the interval \([-\pi, \pi]\) and \( f^{(l)} \) is periodic for each \( 0 \leq l < j \), then
\[
\lim_{{k \to \infty}}k^ja_k(f)  = \lim_{{k \to \infty}} k^jb_k(f) = 0.
\]

    \end{theorem}
\end{frame}



\begin{frame}{Functions of Bounded Variation}
\small
    In order to answer our \emph{Convergence question} we need to have restriction on the \emph{variation} of a function.

    \begin{definition}
    Let $\phi : [a,b] \to \RR$ and $P = \{x_0, \dots, x_n\}$ be an interval partition of $[a,b]$.

Set
\[
V(\phi,P) = \sum_{k=0}^{n-1}|\phi(x_{k+1}) - \phi(x_{k})|.
\]
The \emph{variation} of $\phi$ is defined as
\begin{equation*}
    Var(\phi) \coloneqq \sup\{V(\phi,P)| P \in IP[a,b]\}
\end{equation*}
$\phi$ is said to be \emph{of bounded variation} if $Var(\phi) < \infty$.
\end{definition}

\begin{example}
    The function $\sin(\frac{1}{x})$ is of bounded variations only on the intervals $[a,b]$ which do not contain 0.
\end{example}
    
\end{frame}


\begin{frame}{Relation to Fourier Series}
This theorem will be useful for our answer to convergence question.
    \begin{theorem}\label{growth of ceof. of bdd. var.}
    If $f$ is periodic and of bounded variation, then
    \[
    |ka_k(f)|\leq \frac{Var(f)}{\pi}  \text{ and }  |kb_k(f)| \leq \frac{Var(f)}{\pi} \text{, for all $k \in \NN$.}
    \]
\end{theorem}
\end{frame}

\section{Convergence}

\begin{frame}{Tauberian Type Theorems}
    \begin{theorem}[Tauberian]
    Let $a_k \geq 0$ and $L \in \RR$. If the series $\sum_{k=0}^{\infty} a_k$ is Ceasaro summable to $L$, then it converges to $L$,
    \[
    \sum_{k=0}^{\infty} a_k = L.
    \]
\end{theorem}


\begin{theorem}[Hardy]\label{Hardy}
    Let $E \subset \RR$ and suppose that $f_k: \RR \to \RR$ is a sequence of functions that satisfy
    \[
    |kf_k(x)| < M
    \]
    for all $x \in \RR$, all $k \in \NN$ and some $M > 0$. If $\sum_{k=0}^{\infty}f_k$ is uniformly Cesaro summable to $f$ on $E$,
    then it converges to $f$ uniformly on $E$.
\end{theorem}

\end{frame}


\begin{frame}{Convergence}
    At the end, we reach our aim.
    \begin{theorem}[Dirichlet - Jordan]\label{Dirichlet - Jordan}
    If a real valued function $f$ is periodic of bounded variation on $[-\pi,\pi]$, then
    \[
    \lim_{n \to \infty}S_nf(x) = \frac{f(x+) + f(x-)}{2}
    \]

    If $f$ is also continuous on some closed interval $I$, then
    \[
    \lim_{n \to \infty}S_nf = f
    \]
    uniformly on $I$.
\end{theorem}
\begin{proof}
    \begin{enumerate}
        \item Upper bound for coefficients of functions of bounded variation
        \item Fejer theorem
        \item Hardy theorem
    \end{enumerate}
\end{proof}
\end{frame}





\begin{frame}{Some Examples}

\begin{example}
    Let us demostrate the use of \emph{Dirichlet - Jordan theorem}.
    \begin{itemize}
        \item[(i)] The function $f(x) = x$ can be represented as a series
        \[
        x = 2\sum_{k=1}^{\infty}\frac{(-1)^{k+1}}{k}\sin(kx)
        \]
        whose convergence is strictly pointwise on $(-\pi,\pi)$ and uniformly on any $[a,b] \subset (-\pi,\pi)$.
        \item[(ii)] Since $f(x) = |x|$ is continuous and periodic on $[-\pi,\pi]$, we can use \emph{Dirichlet - Jordan theorem} and conclude that the series
        \[
        |x| = \frac{\pi}{2} - \frac{4}{\pi} \sum_{k=1}^{\infty}\frac{\cos(2k-1)x}{(2k-1)^2}
        \]
        converges uniformly on $[-\pi,\pi]$.
    \end{itemize}
\end{example}
    
\end{frame}



\section{Summary}
\begin{frame}{Summary}
    \small
    \begin{enumerate}
        \item Influences on modern analysis
        \item Trigonometric polynomials and orthogonality properties
        \item Definition of Fourier Series
        \item Cesaro summability of Fourier Series and semi-convergence
        \item Two kernels and their use
        \item Behaviour of Fourier coefficients
        \item Riemann - Lebesque Lemma
        \item Bassel's Inequality and Parseval's Identity 
        \item Functions of bounded variation
        \item Tauberian and Hardy Theorems
        \item Answer to convergence question (Dirichlet-Jordan)
    \end{enumerate}
\end{frame}



\section{References}


\begin{frame}{References}
\small
    \begin{thebibliography}{}

        % Article is the default.
        %\setbeamertemplate{bibliography item}[book]

        \bibitem{Wade}
        Wade, W.R.
        \newblock \emph{Introduction to Analysis}.
        \newblock Pearson, 2004.

        \bibitem{History}
        Niels Jacob and Kristian P. Evans
        \newblock \emph{A course in Analysis}.
        \newblock World Scientific, 2018.

        
    \end{thebibliography}
\end{frame}


\end{document}
